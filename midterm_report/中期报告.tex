\documentclass[11pt,a4paper]{article}

% Compile with XeLaTeX for modern fonts and future multilingual content.

\usepackage{geometry}
\geometry{margin=1in}

\usepackage{graphicx}
\usepackage{booktabs}
\usepackage{amsmath,amssymb}
\usepackage{float}
\usepackage{enumitem}
\usepackage{caption}
\usepackage{subcaption}
\usepackage{hyperref}
\usepackage{xcolor}
\usepackage{microtype}
\usepackage{titlesec}
\usepackage{fancyhdr}
\usepackage{fontspec}

% Fonts (XeLaTeX) — use TeX Live fonts available on Overleaf by default
\setmainfont{TeX Gyre Pagella}
\setsansfont{TeX Gyre Heros}
\setmonofont{TeX Gyre Cursor}

\hypersetup{
  colorlinks=true,
  linkcolor=blue!55!black,
  urlcolor=blue!55!black,
  citecolor=blue!55!black
}

\titleformat{\section}{\Large\bfseries}{\thesection}{0.8em}{}
\titleformat{\subsection}{\large\bfseries}{\thesubsection}{0.8em}{}

\setlist[itemize]{leftmargin=1.25em}

\pagestyle{fancy}
\fancyhf{}
\lhead{Midterm Report: Quantitative Algorithmic Trading System}
\rhead{\thepage}
\renewcommand{\headrulewidth}{0.3pt}

\title{\textbf{Midterm Report: Quantitative Algorithmic Trading System}\\\vspace{0.2em}\large LSTM Baseline (End-to-End Pipeline and Evaluation)}
\author{Liu Chen \quad (22100974D)}
\date{29 Dec 2025}

\begin{document}
\maketitle

\begin{abstract}
This report documents the current stage of the project as an end-to-end, reproducible prototype for model-driven trading research. Using \textbf{2800.HK} daily bars as the primary instrument, we implement: data acquisition, feature engineering, a 3-class labeling scheme with volatility-adaptive thresholds, an LSTM classifier trained via fixed-window walk-forward cross-validation, signal generation with confidence filtering, and a cost-aware backtest on the most recent \textasciitilde1-year out-of-sample period. We present the full workflow, generated artifacts, and the quantitative outcomes from the latest run.
\end{abstract}

\section{Project Context and Objectives}
The overall project scope and phased plan are described in the Proposal (\texttt{Proposal.pdf}). This midterm milestone focuses on building and validating a robust experimental workflow for: (i) predictive modeling on historical data, (ii) translating model outputs into trading decisions, and (iii) evaluating strategy performance under trading frictions (transaction costs).

\section{Reproducible Pipeline (This Run)}
This run is produced by \texttt{run\_lstm2\_pipeline.py}. All artifacts are written to \texttt{outputs/}:
\begin{itemize}
  \item \textbf{Model artifact}: \texttt{outputs/models/lstm2\_last\_fold\_20251229\_183712.pt}
  \item \textbf{Plot}: \texttt{outputs/plots/lstm2\_backtest\_20251229\_183712.png}
  \item \textbf{Reports}: \texttt{outputs/reports/} (run config, CV metrics, CV predictions, backtest stats, and backtest time series)
\end{itemize}

\subsection{Data}
\textbf{Instrument:} 2800.HK (Tracker Fund of Hong Kong). Data is fetched via \texttt{yfinance} with \textbf{auto-adjusted} prices, using parameters \textbf{period=3y} and \textbf{interval=1d}. The run log indicates the effective date range is \textbf{2022-12-29 to 2025-12-29}, with \textbf{733} daily bars.

\subsection{Feature Engineering (16 inputs)}
The model input consists of 16 technical features (see \texttt{outputs/reports/lstm2\_run\_config\_20251229\_183712.json}):
\begin{itemize}
  \item Moving averages: SMA(5, 10, 20, 50)
  \item Rolling variance of 1-day log returns: var(5, 10, 20)
  \item RSI(14)
  \item MACD: line, signal, histogram
  \item Bollinger Bands: mid, upper, lower, bandwidth, \%b
\end{itemize}
Additionally, \texttt{logret\_1d} is computed for labeling but is not included in the 16-dimensional model input.

\subsection{3-Class Labeling (Down / Neutral / Up)}
The prediction target is the next-day log return:
\[
  r_{t+1}=\log\left(\frac{C_{t+1}}{C_t}\right).
\]
A volatility-adaptive threshold is constructed using a 20-day rolling standard deviation of \texttt{logret\_1d} and shifted by one day to avoid future leakage:
\[
  \tau_t=\max(\text{min\_vol},\,k\cdot\sigma_{t}^{(20)}),\quad k=0.8,\ \text{min\_vol}=10^{-4}.
\]
Labels are defined as:
\begin{itemize}
  \item $r_{t+1}\ge \tau_t \Rightarrow$ Up
  \item $r_{t+1}\le -\tau_t \Rightarrow$ Down
  \item otherwise $\Rightarrow$ Neutral
\end{itemize}

\subsection{Sequence Construction (lookback=30)}
Each training sample is a fixed-length sequence of the past \textbf{30} trading days of features, with shape $[30,16]$. The label is aligned to the end of the window.

\section{Model and Training Setup (LSTM Classifier)}
\subsection{Architecture}
We use a PyTorch LSTM classifier with a 3-class softmax output:
\begin{itemize}
  \item LSTM hidden size: 64
  \item Number of layers: 2
  \item Dropout: 0.4
  \item Output: 3 logits (Down/Neutral/Up)
\end{itemize}

\subsection{Loss and Class Imbalance Handling}
Training uses a \textbf{masked cross-entropy} loss: only Up/Down samples contribute to the loss (Neutral is ignored). Class weights for Down vs Up are computed per fold from the training split.

\subsection{Optimization}
\begin{itemize}
  \item Optimizer: Adam (lr=1e-3, weight\_decay=1e-4)
  \item Scheduler: ReduceLROnPlateau (factor=0.5, patience=8, min\_lr=1e-5)
  \item Early stopping: patience=15 on validation loss
  \item Batch size: 64; Max epochs: 100
  \item Device: CPU (this run)
\end{itemize}

\section{Time-Series Cross-Validation (Fixed-Window Walk-Forward)}
We adopt fixed-window walk-forward CV:
\begin{itemize}
  \item Train window: 250 sequence samples
  \item Validation window: 21 samples
  \item Test window: 21 samples
  \item Step size: 21 samples per fold
\end{itemize}
Per fold, standardization statistics (mean/std) are fit on the training split only and then applied to validation/test splits.

\section{From Predictions to Trades: Backtest Logic}
\subsection{Confidence Filtering and Position Rules}
For each out-of-sample day, the model produces probabilities $(P(\text{Down}),P(\text{Neutral}),P(\text{Up}))$. We apply a confidence threshold \textbf{0.57}:
\begin{itemize}
  \item If $\max P < 0.57$, force prediction to Neutral (no trade).
  \item If $P(\text{Up})\ge 0.57$ and $P(\text{Up})\ge P(\text{Down})$, take Long (+1).
  \item If $P(\text{Down})\ge 0.57$ and $P(\text{Down})> P(\text{Up})$, take Short (-1).
\end{itemize}
\textbf{Minimum holding period:} 2 trading days (once a position is opened, it is held for at least 2 days).

\subsection{Returns, Costs, and Equity}
Daily simple return is derived from log return: $R_t=\exp(r_t)-1$. Strategy return is $pos_t\cdot R_t$, and transaction cost is charged on absolute position changes at \textbf{2 bps} per change. Equity curves are computed via cumulative multiplication.

\section{Results of This Run}
\subsection{Backtest Statistics (Last 252 Trading Days)}
Backtest statistics are from \texttt{outputs/reports/lstm2\_backtest\_stats\_20251229\_183712.json} and summarized in Table~\ref{tab:bt}.

\begin{table}[H]
\centering
\caption{Backtest summary (LSTM strategy vs Buy\&Hold, last \textasciitilde1 year)}
\label{tab:bt}
\begin{tabular}{lrr}
\toprule
Metric & LSTM Strategy & Buy\&Hold \\
\midrule
Days & 252 & 252 \\
Total return & -0.0623 & 0.2230 \\
Annualized return & -0.0623 & 0.2230 \\
Annualized volatility & 0.1033 & -- \\
Sharpe ratio & -0.5711 & -- \\
Max drawdown & -0.1344 & -- \\
Position coverage & 0.2540 & 1.0000 \\
Transaction cost & 2.0 bps & -- \\
\bottomrule
\end{tabular}
\end{table}

\subsection{Equity Curve, Position, and Model Confidence}
Figure~\ref{fig:bt} is generated by \texttt{outputs/plots/lstm2\_backtest\_20251229\_183712.png}. The top panel compares strategy equity (after costs) vs Buy\&Hold; the middle panel shows the discrete position; the bottom panel shows $P(\text{Up})$ and $P(\text{Down})$ with the confidence threshold at 0.57.

\begin{figure}[H]
\centering
\includegraphics[width=\textwidth]{lstm2_backtest_20251229_183712.png}
\caption{LSTM pipeline backtest visualization: equity, position, and prediction confidence}
\label{fig:bt}
\end{figure}

\section{Appendix: Run Configuration Snapshot}
The full configuration snapshot is saved in \texttt{outputs/reports/lstm2\_run\_config\_20251229\_183712.json}. Key parameters include: lookback=30; train/val/test=250/21/21; confidence threshold=0.57; minimum holding=2; transaction cost=2 bps.

\end{document}


